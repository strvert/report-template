\documentclass{bxjsarticle}

\usepackage[dvipdfmx]{graphicx}
\usepackage{float}

% プログラムリストの設定
\usepackage{listings}
\lstset{
  basicstyle={\ttfamily},
  identifierstyle={\small},
  commentstyle={\smallitshape},
  keywordstyle={\small\bfseries},
  ndkeywordstyle={\small},
  stringstyle={\small\ttfamily},
  frame={tb},
  breaklines=true,
  columns=[l]{fullflexible},
  numbers=left,
  xrightmargin=0zw,
  xleftmargin=3zw,
  numberstyle={\scriptsize},
  stepnumber=1,
  numbersep=1zw,
  lineskip=-0.5ex
}
\renewcommand{\lstlistingname}{プログラムリスト}


\begin{document}

% 表紙
\begin{center}
  \Huge レポートタイトルI\hspace{-3pt}I \par
  \vspace{15mm}
  \LARGE 実験テーマ \\ \underline{実験テーマ} \par
  \vspace{10mm} 名前\\ \par
  \vspace{40mm}
  \Large 実験日\\0001年01月01日\\
                 0001年01月01日\\
				 0001年01月01日\\
				 0001年01月01日\\
  \vspace{5mm}
         提出日\\ \today \par



  \vspace{15mm}
  \Large 共同実験者\\Alice \\
                     Bob   \\

\end{center}

\thispagestyle{empty}
\clearpage
\addtocounter{page}{-1}

\newpage

% 本文
\section{section}
\subsection{subsection}
\subsubsection{subsubsection}

\section{section}

% プログラム
\begin{lstlisting}[caption=プログラム, label=prog:prog]
int main(int argc, char **argv, char **envp);
if (n % 2 == 0)
	n += 1;
\end{lstlisting}

%\lstinputlisting[caption=プログラム, label=prog:prog]{prog.c}

% 箇条書き
\begin{enumerate}
  \item A
  \item B
  \item C
\end{enumerate}

% 表
\begin{table}[htb]
  \centering
  \caption{表}
  \begin{tabular}{|l|r|}\hline
   \multicolumn{1}{|c|}{OS} & \multicolumn{1}{|c|}{---項目---} \\ \hline \hline
    Windows  & 好き \\  \hline
    Mac      & 好き \\  \hline
    Unix     & 好き \\  \hline

  \end{tabular}
\end{table}

% 図
\begin{figure}[H]
  \centering
  \includegraphics[width=7cm]{image.jpg}
  \caption{画像} \label{fig:image}
\end{figure}

\end{document}




































